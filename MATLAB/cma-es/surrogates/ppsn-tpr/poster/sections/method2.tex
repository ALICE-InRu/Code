
\section{Model Selection}

Two kernel types  are investigated, the
\emph{polynomial kernel}
\begin{equation}
\kappa(\vec{x}_i,\vec{x}_j) = (1 + \inner{x_i}{x_j})^d
\end{equation}
and \emph{Gaussian kernel}
\begin{equation}
\kappa(\vec{x}_i,\vec{x}_j) = \exp\big(-\gamma \norm{\vec{x}_i-\vec{x}_j}^2\big).
\end{equation}

\ \\
When applying kernel methods it is important to scale the
points $\vec{x}$ first. A standard method of doing so is to
scale the training set such that all points are in some range,
typically $[-1,1]$. That is, scaled $\tilde{\vec{x}}$ is
\begin{equation}
\tilde x_i = 2 (x_i - \underline{x}_i) / (\overline{x}_i -
\underline{x}_i) - 1 ~~~ i = 1,\ldots,n
\end{equation}
where $\underline{x}_i$, $\overline{x}_i$ are the maximum and
minimum values of variable $i$ in set $S$.

